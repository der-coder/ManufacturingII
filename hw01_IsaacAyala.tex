\documentclass[a4paper,12pt]{article}

\usepackage{xltxtra}

\usepackage{hyperref}
\hypersetup{
    colorlinks=true,
    linkcolor=black,
    citecolor = black,
%     filecolor=magenta,
    urlcolor=black,
%     pdftitle={<Name>},
%     bookmarks=true
%     pdfpagemode=FullScreen,
}

\usepackage{amsmath,amssymb,amsfonts}
\usepackage{mathtools}

\usepackage{xcolor}
\usepackage{graphicx}
\graphicspath{{img/}}
\usepackage[maxbibnames=9]{biblatex}
\addbibresource{references.bib}


\usepackage{float}
\usepackage[center]{caption}
\usepackage{subcaption}
\usepackage{import}
\usepackage{transparent}
\usepackage{csquotes}

\usepackage{bm}
\usepackage{listings}
    \lstset{
    language=Matlab, % Language
    numbers=left, % where to put the line-numbers
    numberstyle=\footnotesize,% the size of the fonts that are used for the line-numbers
    stepnumber=1,% the step between two line-numbers. If it's 1 each line will be numbered
    numbersep=5pt,% how far the line-numbers are from the code
    showspaces=false,% show spaces adding particular underscores
    showstringspaces=false,% underline spaces within strings
    showtabs=false,% show tabs within strings adding particular underscores
    breaklines=true,% sets automatic line breaking
    breakatwhitespace=false,% sets if automatic breaks should only happen at whitespace
    escapeinside={\%*}{*)}% if you want to add a comment within your code
    }

\newcommand*{\Perm}[2]{{}^{#1}\!P_{#2}}%
\newcommand*{\Comb}[2]{{}^{#1}C_{#2}}%

\usepackage{mathtools}

\DeclarePairedDelimiter\abs{\lvert}{\rvert}%
\DeclarePairedDelimiter\norm{\lVert}{\rVert}%

% Swap the definition of \abs* and \norm*, so that \abs
% and \norm resizes the size of the brackets, and the
% starred version does not.
\makeatletter
\let\oldabs\abs
\def\abs{\@ifstar{\oldabs}{\oldabs*}}
%
\let\oldnorm\norm
\def\norm{\@ifstar{\oldnorm}{\oldnorm*}}
\makeatother

% \usepackage{fancyhdr}
% \pagestyle{fancy}
% \fancyhf{}
% \rhead{}
% \lhead{}
% \fancyfoot[R]{\thepage}


\usepackage[spanish]{babel}
\decimalpoint

\title{Tarea \#1}
\author{Isaac Ayala Lozano}
\date{\today}

\begin{document}
\maketitle

\begin{itemize}
 \item \textbf{Redes neuronales no supervisadas.}
 Se refiere a las redes neurales que furon entrenadas con datos sin depurar o 
sin clasificar por humanos.
La red neuronal es libre de encontrar patrones y asignar relaciones entre los 
datos a su propia manera.
 
 \item \textbf{Redes neuronales supervisadas.} 
 Se consideran así a las redes neuronales que han sido entrenadas con 
información ya clasificada por humanos como material de aprendizaje.
 
 \item \textbf{Reinforcement learning.}
 Estrategia de entrenamiento diferente de las dos anteriores. 
 Emplea una metodología de enseñanza de lazo cerrado, en la que el agente o red 
neuronal interactúa con su entorno para realizar una tarea.
El agente recibe una reñal de recompensa de acuerdo a su desempeño y ajusta sus 
acciones para maximizar dicha señal.
 
 
 \item \textbf{Industria 4.0.}
 La nueva etapa de la industria en el tema de tecnología. La implementación de 
sistemas electrónicos y capaces de conectarse mediante múltiples redes para 
interactuar con el resto de la fábrica. De importancia alta es la habilidad de 
utilizar la red de Internet para controlar dichos dispositivos.
 
\end{itemize}






% \printbibliography

% \newpage
% \pagebreak
% \appendix
% \section{Octave Code}
% \lstinputlisting[language=Matlab]{<filename>.m}

\end{document}
